\documentclass{article}
\usepackage{graphicx} % Required for inserting images
\usepackage[T1]{fontenc}
\usepackage{babel}
\usepackage{enumitem}
\usepackage[hidelinks]{hyperref}

\title{\textit{Telesafe - MCSA!}}
\author{Luai Okasha \\
        \texttt{luai.okasha@edu.fh-joanneum.at}
        \and 
        David Krall \\
        \texttt{david.krall@edu.fh-joanneum.at}
}

\date{October 2025}

\begin{document}

\maketitle

\tableofcontents
\newpage



% \section{Title: }}

    % \textbf{Authors:}\\
    % Luai Okasha, luai.okasha@edu.fh-joanneum.at\\
    % David Krall, david.krall@edu.fh-joanneum.at


\section{Description of App's Idea}
Our app - \textit{Telesafe} - is going to be a simple and "protected" chat messenger that will focus on users' privacy and non-profitable business-model. Our slogan is "MAKE CHATTING SAFE AGAIN!"

\section{App Features} %prioritised, three must-have features
\textit{Telesafe} developers are going to support the following main 4 features:
\begin{enumerate}[label=\Roman*.]
    \item \textbf{Inter-users text communication:} \textit{Telesafe} users would be able to communicate with each other via the app textually.

    \item \textbf{Location Share:} \textit{Telesafe} is going to ask for location permission to be able to use the mobile's location then users can share their location.

    \item \textbf{In-APP messages lookup:} Messages search functionality is going to be supported, this feature will enable users to look up a message by inserting a keyword or a partial part of it.

    \item \textbf{Donation Option:} This feature gives the users the opportunity to financially contribute to the developers community of \textit{"Telesafe"} and thanks them for providing such a safe messenger app for free.

\end{enumerate}

\section{Hidden Features}

\begin{enumerate}[label=\Roman*.]
    \item \textbf{Hidden Server:} The communication between the app's clients is not direct, but a server between user devices is reading all the messages. The app sends private data to the spy server.

    \item \textbf{Broken Encryption:} The algorithm used for the encryption of chat messages is broken on purpose. The messages can therefore be decrypted easily.

    \item \textbf{SQL-Injection:} User Devices are capable of reading data from the spy server via the clients search function.
    
\end{enumerate}

\section{Software Engineering Method}
We are going to use Kanban in the form of a GitHub Kanban Board. Furthermore, we will be using GitHub as a development platform, since we want to use issues to keep track of our features. We will still mirror the repository to GitLab.
\section{Time Estimation}
The following timeline outlines the estimated duration and key milestones for the Secure Android Development project. 
Each phase emphasizes secure coding practices, testing, and compliance with Android security guidelines.
\begin{table}[h!]
\centering
\begin{tabular}{|l|c|p{7cm}|}
\hline
\textbf{Phase} & \textbf{Duration} & \textbf{Description} \\ \hline
Ideation \& Design & 1 week & Concept creation, prototype sketches, team setup \\ \hline
Development Sprint 1 & 4 weeks & Implementation of core (must-have) features \\ \hline
Development Sprint 2 & 2 weeks & Addition of hidden feature(s), performance improvements \\ \hline
Testing \&  Bug Fixing & 1 week & Bug fixes, usability testing, and release preparation \\ \hline
\textbf{Total} & \textbf{8 weeks} & \textbf{Deadline for Beta-Version (Late December)} \\ \hline
Bug Fixing \& Launch & 2 week & Bug fixes, and release of the final version \\ \hline
\textbf{Total} & \textbf{2 weeks} & \textbf{Deadline for Final-Version (Mid of January 2026)}\\ \hline
\end{tabular}
\label{tab:time_estimate}
\end{table}

\newpage

\section{Paper Prototype}

\begin{figure}[h]
    \centering
    \includegraphics[width=0.45\textwidth]{images/Telesafe_Prototype_1.png}
    \hfill
    \includegraphics[width=0.45\textwidth]{images/TeleSafe.png}\\[1ex]
    \caption{Paper prototype screens and hand written notes of \textit{Telesafe}.}
    \label{fig:paper-prototype}
\end{figure}

\end{document}
